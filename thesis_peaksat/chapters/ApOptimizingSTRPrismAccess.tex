\section{Optimizing Access to Alignment prism}
\label{apndx:acces_prism}
The star tracker alignment prism is an extruded orthogonal trianlge shape, with the triangle sides to be 1cm x 1cm, and the extrusion sides to be 1cm. 
To perform a successful optical alignment we need to have an access to at least 2 sides of the alignment prism. From the $+x$ side of the star tracker we have line of sight access to the whole surface of the alignment prism. From the $+y$ side we have to make a cut-out to the PCB of the panels. If the star tracker is placed on the frame, with zero yaw, pitch and roll, then from the y side we will have line of sight to a surface of about $0.3cm^2$. This limit, does depend on the positioning of the star tracker in the frame, and the position of the alignment prism coincides with a part of the frame. If we also imagine that the star tracker will not be positioned perfectly, with its axes aligned with the satellite frame aces, and also that the alignment prism, will not be perfectly mounted to the star tracker (of course it will be characterized) then this $0.3 cm^2$ surface can potentially reduce significantly. 

The solution that we propose is the following: to rotate the star tracker with an angle, that we call $\omega$. Angle $\omega$ can be visualized in \cref{fig:cad02}. We chose for it to have an value of $2.5 \degree$ because that way we can optimize the vertical line of sight that we can have to the alignment prism of the star tracker, without interfering with other components of the satellite. That way, the surface of the alignment prism that we can see vertically, will be $4.6 cm^2$ that is about the same as the surface from the $+x$ side. 

Design of this rotation performed by Mr. Theodoris Balsis, (TODO), and more details can be found in \href{https://ikee.lib.auth.gr/record/368551/files/%CE%98%CE%B5%CF%8C%CE%B4%CF%89%CF%81%CE%BF%CF%82%20%CE%9C%CF%80%CE%AC%CE%BB%CF%83%CE%B7%CF%82%20-%20%E2%88%86%CE%BF%C2%B5%CE%B9%CE%BA%CE%AE%20%CE%B1%CE%BD%CE%AC%CE%BB%CF%85%CF%83%CE%B7%20%CE%BA%CE%B1%CE%B9%20%CF%80%CE%B5%CE%B9%CF%81%CE%B1%C2%B5%CE%B1%CF%84%CE%B9%CE%BA%CE%AE%20%CE%B5%CE%BE%CE%B1%CE%BA%CF%81%CE%AF%CE%B2%CF%89%CF%83%CE%B7%20%CF%84%CE%BF%CF%85%20%CE%BD%CE%B1%CE%BD%CE%BF%CE%B4%CE%BF%CF%81%CF%85%CF%86%CF%8C%CF%81%CE%BF%CF%85%20PeakSat%20%CE%BA%CE%B1%CE%B9%20%CF%84%CF%89%CE%BD%20%CF%85%CF%80%CE%BF%CF%83%CF%85%CF%83%CF%84%CE%B7%C2%B5%CE%AC%CF%84%CF%89%CE%BD%20%CF%84%CE%BF%CF%85.pdf}{his thesis}.


\begin{figure}[h]
    \centering
    \includegraphics[width=0.5\textwidth]{ASAP/media/cad01_st_rot_+x.png}
    \caption{Overview of the rotation of the star tracker, as seen from the +x side of the satellite.}
    \label{fig:cad02}
\end{figure}


