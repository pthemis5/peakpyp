\textgreek{
\chapter*{Περίληψη στα Ελληνικά}
Ο \textlatin{PeakSat} είναι ένας νανοδορυφόρος τριών μονάδων (3Μ), ο οποίος έχει αναπτυχθεί από την ομάδα \textlatin{SpaceDot} στο Αριστοτέλειο Πανεπιστήμιο Θεσσαλονίκης. Ο βασικός σκοπός της αποστολής είναι να επιτελέσει σύζευξη οπτικής επικοινωνίας με έναν Οπτικό Σταθμό Βάσης από Χαμηλή Τροχιά Γης. Αυτό με πιό απλά λόγια σημαίνει πως ο δορυδόρος θα στείλει δεδομένα στη Γη χρησιμοποιόντας διαμορφωμένο \textlatin{laser}, όσο κινείται με μια τεχύτητα της τάξης των $8km /s$. Ο δέκτης του σήματος ονομάζεται Οπτικός Σταθμός Βάσης, και είναι ένα τηλεσκόπιο με έναν δέκτη \textlatin{laser}. Ο Οπτικός Σταθμός Βάσης που θα χρησιμοποιηθεί για τον \textlatin{PeakSat} βρίκεται στον Χολομώντα Χαλκιδικής, και είναι υπό την εποτεία του Εργστηρίου Μηχανικής και Αστροδυναμικής του Αριστοτελείου Πανεπιστημίου. 

Την στιγμή που εγγράφεται το παρόν κείμενο, ο \textlatin{PeakSat} έχει επιτυχώς περάσει τις περιβαλοντικές δοκιμές, και βρίσκεται σε αναμονή προς εκτόξευση. 

Ο βασικός σκοπός αυτής της εργασίας είναι να παρουσιαστεί ένα μικρό κομμάτι δουλειάς που έχει γίνει με τον δορυφόρο, και σχετίζεται άμεσα με το Οπτικό κομμάτι της αποστολής. Για να καταλάβουμε τις μεθόδους που χρησιμοποιήθηκαν, αλλά και την αναγκαιότητά τους θα πρέπει να έχουμε κάποιες πιο γενικές γνώσεις του δορυφόρου. Έγινε προσπάθεια η δουλειά που παρουσιάζεται να είναι διαχωρισμένη σε 5 σχετικά ανεξάρτητα κεφάλαια, προσπαθόντας να διατηρηθεί η γραμμικότητα στην δομή. 

Στο πρώτο κεφάλαιο παρουσιάζουμε κάποιες γενικές πληροφορίες για τον δορυφόρο, τα υποσυστήματά του, τους βασικούς στόχους της αποστολής και τις ιδιότητες της τροχιάς του. Στο 2ο κεφάλαιο συζητάμε τι είναι ένα οπτικό ωφέλιμο φορτίο ενός δορυφόρου, και δίνουμε παραπάνω λεπτομέρειες στο ωφέλιμο φορτίο του \textlatin{PeakSat}. Συνεχίζουμε παρουσιάζοντας μια εισαγωγή στο Υποσύστημα Προσδιορισμου και Ελέγχου Προσανατολισμού (ΥΠΕΠ) του δορυφόρου, δίνοντας βάση στις έννοιες που χρειάζονται για τις οπτικές λειτουργείες του. Στο κεφάλαιο 4, παρουσιάζουμε την μέτρηση γωνιακής απόκλισης που σχεδιάστηκε και πραγματοποιήθηκε από την ομάδα, με σκοπό να βρεθεί ο σχετικός προσανατολισμός του οπτικού ωφέλιμου φορτίου και του ΥΠΕΠ. Τέλος, συνδιάζουμε όσα είπαμε, παρουσιάζοντας τον πρακτικό σχεδιασμό κάποιων οπτικών λειτουργειών της αποστολής. 

Κλείνοντας, θέλουμε να αναφέρουμε πως οι περισσότεροι αλγόριθμοι και κώδικες που χρησιμοποιήθηκαν σε αυτήν την δουλειά είναι διαθέσιμοι στο \textlatin{github}.

 



\chapter*{Ευχαριστίες}
Αυτή η εργασία παρουσιάζει ένα κομμάτι της δουλειάς που έκανα γύρω από τον PeakSat, τον τελευταίο ενάμιση χρόνο. Σε ομαδικό περιβάλλον όπως είναι η κατασκευή ενός δορυφόρου, είναι πολυ δύσκολο να καταφέρω να εκφράσω την ευγνωμοσύνη προσωπικά σε όλα τα άτομα με τα οποία συνεργάστηκα. Θα αναφέρω επιγραμματικά και με χρονολογική σειρά συνεργασίας την Παναγιώτα (που με έβαλε πρακτικά στην ομάδα), τον Παναγιώτη, τον Κίκα, τον Τέο, τον Ανδρόνικο, τον Ηλία, τον Τσούπο και τον Τατς με τους οποίους είχα την χαρά να περάσω το κεφάλαιο του \textlatin{PeakSat}. 


Επίσης θα ήθελα να ευχαριστήσω κάποιους δικούς μου ανθρώπους. Τους γονείς μου, οι οποίοι με στήριξαν όλα αυτά τα χρόνια, και την αδελφή μου. Αξίζει να σημειωθεί πως οι γονείς μου για κάποιο λόγο είχαν αγχωθεί πάρα πολύ τον τελευταίο καιρό αν θα καταφέρω να τελειώσω την πτυχιακή και την σχολή, οπότε πιστεύω θα χαρούν όταν ακούσουν πως έχει παραδωθεί. Ένα μεγάλο ευχαριστώ στους φίλους μου, οι οποίοι με άκουγαν να γκρινιάζω πιο πολύ από κάθε άλλον, και προφανώς στην Βαλεντίνα, που με στήριζε συνεχώς, παραπονιώταν όταν δεν είχα χρόνο για αυτήν, αλλά και με άκουγε υπομονετικά να αλλάζω θέμα πτυχιακής κάθε 2 εβδομάδες. 


Κλείνοντας θα ήθελα να ευχαριστήσω τους υπεύθυνους καθηγητές μου Κλεομένη Τσιγάνη και Ιωάννη Γκόλια, για την καθοδήγηση και υπομονή τους τον τελευταίο ενάμιση χρόνο, γύρω από την δυναμική εξέλιξη και το χάος που (δυστηχώς δεν) επικράτησε γύρω από αυτήν την εργασία. 
}