\chapter*{Preface}

PeakSat is an 3\.U CubeSat, developed by the SpaceDot team in the Aristotle University of Thessaloniki. It's primary Mission Objective is to perform a Free Space Optical Communication Link from Low Earth Orbit (LEO). What this means in simple words, is for the satellite to send a message to the ground, using a laser, while moving at a speed of approximately $8/, km s^{-1}$. The ground receiver of the laser is called an Optical Ground Station (OGS), and is practically a telescope, with a laser receiver. We will perform a detailed discussion on the mission objectives in \cref{sec:mission_objectives}. 

At the time of writing, PeakSat has successfully passed all the testing phase, and is currently waiting to be launched in a closed box. 
TODO: maybe add some more about what has been done in general, and work that has been done before you joined the team. 
\textit{In this thesis, our main objective is to present the work that has been done with the satellite, and is directly related to the Optical Operations.}
In order to understand the methods discussed and their necessity, we need a more spherical knowledge of the satellite. So, we have tried to split the work and discussed in this thesis in 5 relatively independent parts. We tried to keep a linear presentation structure, starting with the satellite introduction, then presenting the Optical Payload and the ADCS, showcasing the alignment measurement performed to bridge the gap between the 2, and finally combining all the above together to construct in-orbit scenarios. However, there are some section, especially in the ADCS chapter, that will require knowledge of the Alignment methods to be fully understood. 

To explain in more detail the structure of this work, we start in \cref{chapt:introduction} by introducing the basics of the satellite, it's components the mission objectives, and finally the basic properties of it's orbit, and how can we use them for the optical operations. Then, we continue in \cref{chapt:optical_payload} by discussing what an optical payload is, some specific details about PeakSat's optical Payload and testing that we have performed to it's Optical Components. Subsequently in \cref{chapt:adcs} we perform a similar introduction to Attitude Determination and Control Subsystem (ADCS), discuss the ADCS of PeakSat and how to configure it. We close \cref{chapt:adcs} by discussing configuration performance simulations that were performed, and explain why are these critical for performing a successful optical pass. In \cref{chapt:alignment}, we explain in detail the Misalignment Measurement experiment that we designed in order to be able to characterize the Angular Divergence between our ADCS and Optical Payload, and showcase this measurement's importance. Finally, we close this work in \cref{chapt:operations}, by bridging all the aforementioned topics into creating flight representative Mission Tests, as well as real Operations Optical Pass planning. 

Several algorithms and procedures explained here have been programmed into simple \texttt{python} notebooks. More details about all the codes used can be found in \cref{sec:codes_used}, and if a notebook or script exists for a specific section, it will be explicitly mentioned inside it. 